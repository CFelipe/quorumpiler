\documentclass[12pt, a4paper]{article}
\usepackage[utf8]{inputenc}
\usepackage[brazilian]{babel} % Hifenização e dicionário
\usepackage[left=3.00cm, right=2.00cm, top=3.00cm, bottom=2.00cm]{geometry}
\usepackage{enumitem} % Para itemsep etc
\usepackage{longtable} % Dependência do longtabu
\usepackage{tabu} % Para melhor criação de tabelas
\usepackage{listings} % Para códigos
\usepackage{lstautogobble} % Códigos indentados corretamente
\usepackage{color} % Para coloração de códigos
\usepackage{zi4} % Para fonte de códigos
\usepackage{parskip} % Linha em branco entre parágrafos em vez de recuo
\usepackage{graphicx}
\usepackage{verbatim} % Para comentários
\usepackage[breaklinks]{hyperref}

\tabulinesep=0.75ex % Espaçamento interno das tabelas

\definecolor{bluekeywords}{rgb}{0.13, 0.13, 1}
\definecolor{greencomments}{rgb}{0, 0.5, 0}
\definecolor{redstrings}{rgb}{0.9, 0, 0}
\definecolor{graynumbers}{rgb}{0.5, 0.5, 0.5}
\definecolor{graybox}{rgb}{0.9, 0.9, 0.9}

\hypersetup{
    colorlinks=true,
    linkcolor=blue,
}

\newcommand{\ic}[1]{\textbf{\lstinline{#1}}}

\usepackage{listings}
\lstset{
    autogobble,
    columns=fullflexible,
    showspaces=false,
    keepspaces=true,
    showtabs=true,
    breaklines=true,
    showstringspaces=false,
    breakatwhitespace=true,
    escapeinside={(*@}{@*)},
    commentstyle=\color{greencomments},
    keywordstyle=\color{bluekeywords},
    stringstyle=\color{redstrings},
    numberstyle=\color{graynumbers},
    basicstyle=\ttfamily\footnotesize,
    frame=l,
    framesep=12pt,
    xleftmargin=12pt,
    tabsize=4,
    captionpos=b,
}

\begin{document}
\begin{center}
    \textsc{Universidade Federal do Rio Grande do Norte} \\
    \textsc{Departamento de Informática e Matemática Aplicada}
\end{center}

\bigskip

\begin{tabular}{@{}ll@{}}
    \emph{Disciplina:} & DIM0611 --- Compiladores \\
    \emph{Docente:}    & Martin Alejandro Musicante \\
    \emph{Discente:}   & Felipe Cortez de Sá \\
\end{tabular}

\bigskip

\begin{center}
\large \textbf{Relatório} --- Analisador léxico
\end{center}

\section{A linguagem}
\subsection{Tipos}
\subsubsection{\ic{integer}}
Um número positivo, negativo ou zero, ou seja, inteiro. Decidiu-se tratar o
operador unário \ic{-} como token separado do número. Assim, a
expressão regular para um inteiro é simplesmente \ic{[0-9]+}.

\subsubsection{\ic{number}}
Um número em ponto flutuante. Não há suporte a notação científica. Dessa forma,
a expressão regular associada ao token é \ic{[0-9]+"."[0-9]*}

\subsubsection{\ic{boolean}}
Pode assumir apenas os valores \ic{true} ou \ic{false}. Logo, para o token
booleano, temos a expressão regular \ic{\"true\"|\"false\"}

\subsubsection{\ic{text}}
Uma string. Como é não-trivial tratar as quebras de linha e o que deve vir
dentro das aspas, foi retirada
\href{http://www.lysator.liu.se/c/ANSI-C-grammar-l.html}{deste site} uma
expressão regular para tokens de strings. A expressão regular extraída e
adaptada é \ic{\\\"(\\\\.|[\^\\\\\"])*\\\"}

Para a \emph{BNF}, temos então os tipos \ic{type ::= \"integer\" | \"number\" | \"boolean\" | \"text\"}

\subsubsection{Conversão entre tipos}
A conversão entre tipos se dá através da função \ic{cast(tipo, identificador)}.

\subsubsection{Identificadores}
Devem começar com uma letra seguida por zero ou mais combinações entre números, letras
ou underscores. Temos, então, a expressão regular
\ic{([a-zA-Z])([a-zA-Z]|[0-9]|"_")*}

\subsubsection{Declaração e atribuição}
A declaração possui a forma \ic{tipo identificador = valor}. A atribuição
inicial é obrigatória, portanto a BNF associada à declaração é \ic{type ID
\"=\" expr}. A atribuição possui a forma \ic{identificador = valor},
portanto a BNF associada é \ic{ID \"=\" expr}

\subsection{Operadores}
\subsubsection{\ic{+}}
Soma números e concatena strings. Funciona apenas como operador binário, não
sendo possível escrever \ic{+5 - 3}, por exemplo.

\subsubsection{\ic{-}}
Subtrai números e funciona também como operador unário, sendo possível escrever
\ic{-5 - 3}.

\subsubsection{\ic{*}, \ic{/}}
Multiplicação e divisão entre \ic{integers} e \ic{numbers}.

\subsubsection{\ic{mod}}
Retorna o resto de uma operação de divisão. Funciona também para o tipo
\ic{number}, não se restringindo aos inteiros.

\subsubsection{\ic{>}, \ic{<}, \ic{>=}, \ic{<=}, \ic{==} e \ic{not=}}
Operadores relacionais

\subsection{Comentários}
Os comentários são definidos semelhantemente a \emph{C++} e \emph{Java}, ou
seja, \ic{//} é usado para comentários de linha e \ic{/* */} é usado para
comentários multilinha. Visto que identificar comentários pelo analisador
léxico é não-trivial, foram utilizadas as expressões regulares
\href{http://stackoverflow.com/a/25396611}{deste site}: \ic{\"//\".*} para
comentários de linha e \ic{[/][*][^*]*[*]+([^*/][^*]*[*]+)*[/]} para
comentários multilinha. Adicionalmente, a detecção de comentários não
terminados é feita através da expressão \ic{[/][*]}.

\subsection{Estruturas de controle}
\subsubsection{if}
Executa um bloco de instruções caso a expressão especificada após a palavra
\emph{if} seja avaliada como verdadeira. A BNF associada é \ic{\"if\" expr
\{stmt\} \{\"elseif\" expr \{stmt\}\} [\"else\" expr \{stmt\}] \"end\"}

\subsubsection{repeat \emph{n} times}
Repete um bloco de instruções \emph{n} vezes, sendo \emph{n} qualquer número,
inclusive negativo ou zero (que nada executam)
\subsubsection{repeat while}
Repete um bloco de instruções enquanto uma determinada condição especificada
depois da palavra \ic{while} for avaliada como verdadeira.
\subsubsection{repeat until}
Repete um bloco de instruções enquanto uma determinada condição especificada
depois da palavra \ic{while} for avaliada como falsa.

Desta forma, a BNF para as estruturas de repetição é \ic{\"repeat\" ((\"while\"
| \"until\") expr | NUMBER times) \"end\"}

\subsection{Actions}
Um bloco de instruções que, após ser definido, pode ser chamado em qualquer
parte do programa. % definição ruim?
Recebe zero ou mais parâmetros e opcionalmente retorna um valor de um tipo
especificado na definição da ação. Temos a BNF \ic{\"action\" ID [\"(\" args_def
\")\"] [\"returns\" type] {stmt} \"end\"}

\subsection{Classes}
De acordo com a documentação, uma classe representa uma coleção de dados e
ações sobre esses dados. Para definir uma nova classe, utiliza-se \ic{class
identificador end}. Apenas declarações de variáveis, atribuições e ações podem
aparecer dentro desse bloco.

\subsubsection{Herança}
Opcionalmente, uma classe pode herdar atributos e ações de outras classes. Isso
é feito escrevendo \ic{class identificador is identificadores}, em que
\ic{identificadores} pode conter uma ou mais classes separadas por vírgula.

Assim, a BNF para declaração de classes é \ic{\"class\" ID [\"is\" ID \{\",\"
ID\}] \{declaration | action\} \"end\"}

\subsection{Classes genéricas}
São definidas como uma classe comum, porém utilizando a forma \ic{class
Identificador<Type>} e instanciadas especificando o tipo desejado:
\ic{Identificador_da_classe<tipo> identificador}

\subsection{Autoboxing}
\emph{Autoboxing} é o nome dado à conversão automática de tipos primitivos
realizada ao utilizar classes genéricas.

\subsection{Tratamento de erros}
Para blocos \ic{try-catch-finally} comumente encontrados em outras linguagens,
são utilizadas as palavras \ic{check-detect-always}.

\subsection{Estrutura de um programa}
É necessário colocar chamadas de ações dentro de um ação especial \ic{Main},
opcionalmente localizada dentro de uma classe \ic{Main}. Fora da classe
principal, é possível declarar variáveis, realizar atribuições, definir
estruturas de controle e classes e realizar detecção de erros.

\subsection{BNF completa}
\begin{lstinputlisting}{../quorum.bnf}
\end{lstinputlisting}

\end{document}
